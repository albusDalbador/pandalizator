\documentclass[polish,12pt]{article}

\usepackage{graphicx}
\graphicspath{{./images/}}

\usepackage{geometry}
\geometry{left=15mm,
          right=15mm,
          top=20mm}

\usepackage{caption}

\usepackage{amsmath}

\title{Projekt zaliczeniowy\\Analiza obrazów}
\author{Mikhail Lemiasheuski\\Mateusz Nogaj\\Rafał Gonet}


\begin{document}

\maketitle


\section{Opis projektu}
Celem projektu jest stworzenie aplikacji rozpoznającej pandy na podanym obrazku. I.e. wyświetlenie informacji o tym, czy na podanym obrazku znajduje się kształt pandy. W celu komunikacji z użytkownikiem opracowany został interfejs graficzny, umożliwiający użytkownikowi podstawową komunikację z programem (załadowanie/podanie ścieżki do pliku oraz wyświetlenie wyniku)

\section{Założenia wstępne}
W wersji końcowej stworzona została następująca funkcjonalność:
\begin{itemize}
    \item Proekt pozwala na załadowanie obrazka w każdym formacie, wspieranym przez bibliotekę OpenCV, są to między innymi - .JPG, JPEG, .PNG, .BMP
    \item Wielki przycisk "Sprawdz", uruchamiający cały kod programu
    \item Wyświetlenie wyniku analizy obrazka (obrazek w raz z informacją o obecności na nim pandy)
\end{itemize}

\pagebreak
\section{Opis rozwiązania}

\subsection{Ogólny zarys metody rozwiązania}
Rozpoznanie kształtu pandy oparliśmy głownie o skład kolorowy obrazka oraz kształty na nim obecne. Zauważyliśmy, że histogram rozkładu wartości pikseli 0-255 obrazka z panda w skali szarości ma dosyć harakterystyczną postać 
\begin{center}
    \includegraphics*[scale=0.4]{images/panda_hist.png}
\end{center}
Jednym z parametrów modelu wybraliśmy stosunek 50 pikseli "najbardziej czarnych" do ilości nastpępnych 50 "mniej czarnych" w celu wyznaczenia, czy istnieje na histogramie obrazka pokazany "szczyt".
\begin{center}
    \includegraphics*[scale=0.4]{images/panda_hist_proportion.png}
\end{center}

Drugim parametrem była liczba Eulera dla zbinaryzowanego i odszumowanego obrazka. Celem tego parametru jest pokazanie ilości czarnych plam na obrazie, dlatego obrazek jest binaryzowany niskim progiem - 25/255.

W celu uzyskania bardziej dokładnej wartości danego parametru w programie są stosowane algorytmy domknięcia oraz otwarcia binarnego dostarczanych przez bibliotekę \textit{SciPy} oraz funkcja \textit{findContours} z biblioteki \textit{OpenCV} do obliczenia bezpośrednio liczby Eulera.

\subsection{Perceptron}
Jądrem programu jest klasa \textit{Perceptron} implementująca podstawowe funkcje do modyfikacji wag, zapisania ich do pliku oraz testowania "nauczonego" modelu. 

W wersji "realisowej" klasa jest inicjowana wczytanymi z pliku .json wartościami wag w celu przyśpieszenia uruchomienia aplikacji.


\subsection{Zbiór danych uczących}



\subsection{Testowanie modelu}


\section{Podsumowanie, analiza błedów}

Zastosowana w projekcie metoda uczenia maszynowego nie odpowiada celom działania programu ze względu na poniższe fakty:

\begin{itemize}
    \item zbiór danych na których opiera się proces "uczenia się" algorytmu nie jest podzielny liniowo w żadnym z wymiarów. I.e. 
    \item wybrane kryteria rozpoznawania pandy są "lekko" prymitywne 
\end{itemize}

W trakcie pracy nad projektem dowiedzieliśmy się o istnieniu modelu \textbf{MLP (multilayer perceptron)} używanego do tworzenia algorytmów klasyfikacji obrazów. Ze względu na poziom skomplikowania podobnego rozwiązania i ograniczony czas na stworzenie projektu pozostaliśmy przy opisanym we wcześniejszej części prostym algorytmie.

\section{Podział pracy i analiza czasowa}

\end{document}